\documentclass[11pt]{article}
\usepackage{graphicx} % This lets you include figures
\usepackage{hyperref} % This lets you make links to web locations
\usepackage[margin=0.5in]{geometry}
\usepackage[rightcaption]{sidecap}
\usepackage{subcaption}
\usepackage{wrapfig}
\usepackage{float}
\usepackage{imakeidx}
\usepackage{indentfirst}
\makeindex
%---------------------------Do Not Edit Anything Above This Line!!------------------------

% edit the line below, if needed, to change the directory name for your image files.
\graphicspath{ {./images/} }



\begin{document}

%---------------------------Edit Content in the Box to Create the Title Page--------------
\begin{titlepage}
   \begin{center}
       \vspace*{1cm}
	   \Huge
       \textbf{Project Title}

       \vspace{0.5cm}
       \Large
       Sprint Number \\
       Date \\
   \end{center}

       \vspace{1.5cm}

\begin{table}[h!]
\centering
\begin{tabular}{|l|l|}
\hline
\textbf{Name} & \textbf{Email Address} \\ \hline
Name1         & WKU email address1         \\ \hline
Name2         & WKU email address2         \\ \hline
\end{tabular}
\end{table}

%Latex Table Generator    
%https://www.tablesgenerator.com/     
        
\vspace{4in}

\centering        
CS 396 \\
Fall 2025\\
Project Technical Documentation

\end{titlepage}
%---------------------------Edit Content in the Box to Create the Title Page--------------


% No text here.


%---------------------------Do Not Edit Anything In This Box!!------------------------
%Table of contents and list of figures will be autogenerated by this section.
\newpage
\setcounter{page}{1}%
\cleardoublepage
\pagenumbering{gobble}
\tableofcontents
\cleardoublepage
\pagenumbering{arabic}
\clearpage
\newpage
\setcounter{page}{1}%
\cleardoublepage
\pagenumbering{gobble}
\listoffigures
\cleardoublepage
\pagenumbering{arabic}
\newpage
%---------------------------Do Not Edit Anything In This Box!!------------------------




%---------------------------Project Introduction Section------------------------------

% No text here.

\section{Introduction} %\section{} is used to create major section headers

% No text here.

%---------------------------Project Overview------------------------------------------
\subsection{Project Overview} %\subsection{} is used to create minor sections 
% 300 words
% Description of the project, what the project provides, its purpose, problems solved, and target audience.

Lorem ipsum dolor sit amet, consectetur adipiscing elit, sed do eiusmod tempor incididunt ut labore et dolore magna aliqua. Aliquet lectus proin nibh nisl condimentum id. Lorem dolor sed viverra ipsum nunc aliquet. Magna fringilla urna porttitor rhoncus dolor. Bibendum at varius vel pharetra vel turpis nunc eget. Fermentum posuere urna nec tincidunt praesent semper. 
%use blank lines to begin a new paragraph

Nunc congue nisi vitae suscipit tellus mauris a. Tellus at urna condimentum mattis pellentesque id nibh. Massa tincidunt dui ut ornare lectus. Quisque id diam vel quam elementum. Nunc lobortis mattis aliquam faucibus. Tellus elementum sagittis vitae et. Eget felis eget nunc lobortis mattis aliquam faucibus purus. 

%---------------------------End Project Overview---------------------------------------

% No text here.

%---------------------------Project Scope----------------------------------------------
\subsection{Project Scope}
% 350 words
% Description of all deliverables, benefits, outcomes, and work required (all tasks, costs, time, people, resources, dates/deadlines, and final deliverables date).

Text goes here.

%---------------------------End Project Scope---------------------------------------

% No text here.


\subsection{Technical Requirements}


%---------------------------Functional Requirements----------------------------------------------
\subsubsection{Functional Requirements} %\subsubsection{} used to create sections for parent subsections.
% Functional requirements define what a system or software must do, specifying the desired behavior or functionality.

% List as atomic bullet points that can be tested

\begin{table}[h!]
\centering
\begin{tabular}{|l|}
\hline
\textbf{Mandatory Functional Requirements} \\ \hline
Req 1                                      \\ \hline
Req 2                                      \\ \hline
Req 3                                      \\ \hline
                                           \\ \hline
                                           \\ \hline
\textbf{Extended Functional Requirements}  \\ \hline
Ext. Req 1                                 \\ \hline
Ext. Req 2                                 \\ \hline
Ext. Req 3                                 \\ \hline
                                           \\ \hline
                                           \\ \hline
\end{tabular}
\end{table}

% Paragraph (150 words) explaining the need and purpose for the listed Functional Requirements.
Text goes here.


%---------------------------End Functional Requirements----------------------------------------------

% No text here.

%---------------------------Non-Functional Requirements----------------------------------------------
\subsubsection{Non-Functional Requirements}
% Non-functional requirements specify the constraints, qualities, or attributes that the system or software must possess, such as performance, security, usability, portability, fault tolerance, or reliability.

% List as atomic bullet points that can be tested

\begin{table}[h!]
\centering
\begin{tabular}{|l|}
\hline
\textbf{Mandatory Non-Functional Requirements} \\ \hline
Req 1                                      \\ \hline
Req 2                                      \\ \hline
Req 3                                      \\ \hline
                                           \\ \hline
                                           \\ \hline
\textbf{Extended Non-Functional Requirements}  \\ \hline
Ext. Req 1                                 \\ \hline
Ext. Req 2                                 \\ \hline
Ext. Req 3                                 \\ \hline
                                           \\ \hline
                                           \\ \hline
\end{tabular}
\end{table}

% Paragraph (150 words) explaining the need and purpose for the listed Non-Functional Requirements.
Lorem ipsum dolor sit amet, consectetur adipiscing elit, sed do eiusmod tempor incididunt ut labore et dolore magna aliqua. Aliquet lectus proin nibh nisl condimentum id. Lorem dolor sed viverra ipsum nunc aliquet. Magna fringilla urna porttitor rhoncus dolor. Bibendum at varius vel pharetra vel turpis nunc eget. Fermentum posuere urna nec tincidunt praesent semper. 
%use blank lines to begin a new paragraph

Nunc congue nisi vitae suscipit tellus mauris a. Tellus at urna condimentum mattis pellentesque id nibh. Massa tincidunt dui ut ornare lectus. Quisque id diam vel quam elementum. Nunc lobortis mattis aliquam faucibus. Tellus elementum sagittis vitae et. Eget felis eget nunc lobortis mattis aliquam faucibus purus. Risus commodo viverra maecenas accumsan lacus vel facilisis. Nullam vehicula ipsum a arcu cursus vitae. Morbi tristique senectus et netus et malesuada.
%use blank lines to begin a new paragraph

Lorem ipsum dolor sit amet, consectetur adipiscing elit, sed do eiusmod tempor incididunt ut labore et dolore magna aliqua. Aliquet lectus proin nibh nisl condimentum id. Lorem dolor sed viverra ipsum nunc aliquet. Magna fringilla urna porttitor rhoncus dolor. Bibendum at varius vel pharetra vel turpis nunc eget. Fermentum posuere urna nec tincidunt praesent semper. 
%use blank lines to begin a new paragraph

Nunc congue nisi vitae suscipit tellus mauris a. Tellus at urna condimentum mattis pellentesque id nibh. Massa tincidunt dui ut ornare lectus. Quisque id diam vel quam elementum. Nunc lobortis mattis aliquam faucibus. Tellus elementum sagittis vitae et. Eget felis eget nunc lobortis mattis aliquam faucibus purus. Risus commodo viverra maecenas accumsan lacus vel facilisis. Nullam vehicula ipsum a arcu cursus vitae. Morbi tristique senectus et netus et malesuada.
%use blank lines to begin a new paragraph


%---------------------------End Non-Functional Requirements---------------------------------------

% No text here.



%---------------------------DevOps - Continuous Delivery Approach and Results-------------------------------------------------

\section{DevOps - Continuous Integration and Continuous Delivery Approach and Results}
%describe your set of practices that enable you to release software quickly, safely, and sustainably by automating the software delivery process, resulting in faster time to market, improved software quality, and enhanced team productivity.



%---------------------------End DevOps - Continuous Delivery Approach and Results-------------------------------------------------



%---------------------------DevOps - Architecture Approach, Models, and Results-------------------------------------------------

\section{DevOps - Architecture Approach, Models, and Results}
%describe how you designed systems with loosely coupled components, allowing you to work independently, deploy frequently, and scale efficiently, ultimately enabling rapid delivery of software and system reliability.




%---------------------------End DevOps - Architecture Approach, Models, and Results-------------------------------------------------



% No text here.

%---------------------------DevOps - Product and Process Approach and Results-------------------------------------------------

\section{DevOps - Product and Process Approach and Results}
%describe your focus on creating customer-centric, feedback-driven development processes that integrate cross-functional teams, enabling continuous improvement and alignment between product development and business outcomes.




%---------------------------End DevOps - Product and Process Approach and Results-------------------------------------------------

% No text here.


%---------------------------DevOps - Product Management and Monitoring Approach and Results-------------------------------------------------

\section{DevOps - Product Management and Monitoring Approach and Results}
%describe how you implemented proactive, data-driven monitoring and feedback systems that provide real-time insights into system performance, enabling you to identify issues early, improve decision-making, and optimize both system stability and team performance.




%---------------------------End DevOps - Product Management and Monitoring Approach and Results-------------------------------------------------


% No text here.

%---------------------------Product DevOps - Cultural Approach and Results-------------------------------------------------

\section{DevOps - Cultural Approach and Results}
%describe how your team created a collaborative, trust-based environment that emphasizes shared responsibility, continuous learning, and experimentation, driving high performance and innovation.




%---------------------------End DevOps - Cultural Approach and Results-------------------------------------------------


% No text here.

%---------------------------Software Product Testing Section-------------------------------------
\section{Software Testing and Results}



%---------------------------Software Testing Plan Template-------------------------------------

\subsection{Software Testing Plan Template}
%Each of the testing levels (unit, Integration, System, Acceptance) should use the following test plan template.

\textbf{Test Plan Identifier:} %Provides a unique identifier for the test. Every test should have a unique identification number for reference.

\textbf{Introduction:} % 50 words. Brief description and objective about the test type.

\textbf{Test item:} %50 words. Includes detailed information about the Software Under Test (SUT).

\textbf{Features to test/not to test:} %50 words. In scope features. This could be newly added or updated features. Out of scope features not tested. [Provide reasoning for exclusion, like, non-impacted, low priority, etc.]

\textbf{Approach:} %50 words. Strategy to test the software. Includes types of tests and how to test. Functional, performance, security testing using combined [manual + automation], manual only, automation only approach.

\textbf{Test deliverables:} %50 words. All the deliverables from the testing e.g. approaches, test cases, reports etc.

\textbf{Item pass/fail criteria:} %50 words. Entry and Exit criteria for all items. 

\textbf{Environmental needs:} %50 words. Infrastructure required for SUT and executing test cases.

\textbf{Responsibilities:} %50 words. Roles and responsibilities for various testing / supported activities.

\textbf{Staffing and training needs:} %50 words. Training needs to bridge the gap of available and expected skill.

\textbf{Schedule:} %50 words.  Test schedule should also be noted in the Gantt Chart. Test estimation (Efforts) and high-level schedule. Schedule should be for key deliverables or important milestones. Ideally, all test deliverables included in the test plan should be scheduled.

\textbf{Risks and Mitigation:} %100 words. Risk identification for applicable items, assumptions, and mitigation plan.

\textbf{Approvals:} %Approvals and sign of dates.

%---------------------------Software Testing Plan Template-------------------------------------





%---------------------------End Software Product Testing Section-------------------------------------


% No text here.



%---------------------------Conclusion Section-------------------------------------
\section{Conclusion}
%200 words
%Concluding remarks that summarizes the purpose and outcomes of the technical document.  Discussion of short comings and future work.
Text goes here.

%---------------------------End Conclusion Section-------------------------------------


% No text here.



%---------------------------Appendix Section-------------------------------------------
\section{Appendix}

\subsection{Software Product Build Instructions}
%Include in this section all steps for copying the current state of the product to new computers for continued development.
Text goes here.

\subsection{Software Product User Guide}
%Include in this section an overview guide on how to use your software product for a general user and an administrative user.
Text goes here.

\subsection{Source Code with Comments}
%Include in this section all final source code for the product.  Label each file with headings such as, C.1 file1.c, C.2 file2.c, C.3 file1.py, etc.  All source code should be effectively commented.
Text goes here.







%---------------------------End Appendix Section-------------------------------------------














%example image:  uncomment to show usage
%\begin{figure}[h]
%    \centering
%    \includegraphics[width=1\textwidth]{images/Add_non-music.png}
%    \caption{This is how you add non-music items.}
%    \label{fig16}
%\end{figure}


%example links:  uncomment to show usage.
%\url{https://www.youtube.com}
%\href{https://www.wku.edu/}{WKU Homepage}
%\footnote{You can put the link in a footnote like this.}

% Anything to the right of a percent sign will be ignored by LaTeX.
% You can use this to put notes to yourself.  



\end{document}
